% !Mode:: "TeX:UTF-8"

%%% 此部分包含: (1) 学位论文独创性声明 (无需改动) ; (2) 学位论文版权使用授权书 (无需改动).

\thispagestyle{empty}
\renewcommand{\baselinestretch}{1.5}  %下文的行距
%%% -------------  学位论文独创性声明 (无需改动)-------------   %%%
\vspace*{20pt}
\begin{center}{\songti \xiaosanhao \textbf{学位论文独创性声明}}\end{center}
\par\vspace*{30pt}

{\xiaosihao 本学位论文是作者在导师的指导下进行的研究工作及取得的研究成果。据我们所知,除文中已经注明引用的内容外,本论文不包含其他个人已经发表或撰写过的研究成果。对本文的研究做出重要贡献的个人和集体,均已在文中作了明确说明并表示谢意。学位论文作者和导师均承担本声明的法律责任。}\\

\begin{flushright}


\xiaosihao
	\begin{tabular}{lp{3cm}lp{3cm}}
  	学位论文作者签名: & \hfill & 日期:& \hfill \\ \cline{2-2}\cline{4-4}
	导 \hfill 师 \hfill 签 \hfill 名: & \hfill & 日期:& \hfill \\ \cline{2-2}\cline{4-4}
	\end{tabular}
\end{flushright}

%%% -------------  学位论文独创性声明 (无需改动)-------------   %%%
\vspace*{56pt}

\begin{center}{\songti \xiaosanhao \textbf{学位论文版权使用授权书}}\end{center}
\par\vspace*{30pt}

{\xiaosihao 本学位论文作者完全了解淮北师范大学有关保留、使用学位论文的规定,即:研究生在校攻读学位期间论文工作的知识产权单位属淮北师范大学。学校有权保留并向国家有关部门或机构送交论文的复印件和电子版,允许论文被查阅和借阅。本人授权淮北师范大学可以将本学位论文的全部或部分内容编入有关数据库进行检索。可以采用影印、缩印或扫描等复制手段保存、汇编本学位论文。保密的学位论文在解密后适用本授权书。}\\

\begin{flushright}\xiaosihao
	\begin{tabular}{lp{3cm}lp{3cm}}
  	学位论文作者签名: & \hfill & 日期:& \hfill \\ \cline{2-2}\cline{4-4}
	导 \hfill 师 \hfill 签 \hfill 名: & \hfill & 日期:& \hfill \\ \cline{2-2}\cline{4-4}
	\end{tabular}
\end{flushright}
