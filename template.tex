% !Mode:: "TeX:UTF-8"
%% 请使用 XeLaTeX 编译本文. 默认使用 Adobe 字体. 需安装 Adobe fonts 文件夹中的字体.

\documentclass[forlib]{HNUthesis}   % 选项 forprint: 交付打印时, 建议加上此选项, 以消除彩色链接文字, 避免彩色字迹打印偏淡.
                                    % 选项 forlib: 提交给图书馆的电子版, 需要加上选项 forlib, 以消除空白页和彩色链接.

\bibliographystyle{abbrv}           % 参考文献样式,  plain,unsrt,alpha,abbrv 等等

\graphicspath{{figures/}}           % 图片放在 figures 文件夹.


\begin{document}

\fenleihao{O159}  % 分类号
\xuexiaodaima{10486}   % 学校代码
\miji{绝密}           % 密级
\xuehao{201408193}    % 学号

\title{相对拓扑的若干性质研究}               % 论文名
\Etitle{A \LaTeX~Thesis Template for HNU}   % 英文题目
\author{高淑红}                             % 作者名
\Csupervisor{卢涛\quad 教授}                % 指导教师中文名、职称
\Cmajor{计算数学}                           % 专业中文名
\Especiality{格上拓扑学}                    % 研究方向
\date{二〇一五年三月}                       % 日期

\maketitle

% !Mode:: "TeX:UTF-8"

%%% 此部分包含: (1) 学位论文独创性声明 (无需改动) ; (2) 学位论文版权使用授权书 (无需改动).

\thispagestyle{empty}
\renewcommand{\baselinestretch}{1.5}  %下文的行距
%%% -------------  学位论文独创性声明 (无需改动)-------------   %%%
\vspace*{20pt}
\begin{center}{\songti \zihao{-3} \textbf{学位论文独创性声明}}\end{center}
\par\vspace*{30pt}

{\fangsong\zihao{-4} 本学位论文是作者在导师的指导下进行的研究工作及取得的研究成果。据我们所知,除文中已经注明引用的内容外,本论文不包含其他个人已经发表或撰写过的研究成果。对本文的研究做出重要贡献的个人和集体,均已在文中作了明确说明并表示谢意。学位论文作者和导师均承担本声明的法律责任。}\\

\begin{flushright}


\fangsong\zihao{-4}
	\begin{tabular}{llll}
  	学位论文作者签名: & \underline{\makebox[3cm][l]{\hfill}} & 日期:& \underline{\makebox[3cm][l]{\hfill}} \\
		导 \hfill 师 \hfill 签 \hfill 名: & \underline{\makebox[3cm][l]{\hfill}} & 日期:& \underline{\makebox[3cm][l]{\hfill}} \\
	\end{tabular}
\end{flushright}

%%% -------------  学位论文独创性声明 (无需改动)-------------   %%%
\vspace*{56pt}

\begin{center}{\songti \zihao{-3} \textbf{学位论文版权使用授权书}}\end{center}
\par\vspace*{30pt}

{\fangsong\zihao{-4} 本学位论文作者完全了解淮北师范大学有关保留、使用学位论文的规定,即:研究生在校攻读学位期间论文工作的知识产权单位属淮北师范大学。学校有权保留并向国家有关部门或机构送交论文的复印件和电子版,允许论文被查阅和借阅。本人授权淮北师范大学可以将本学位论文的全部或部分内容编入有关数据库进行检索。可以采用影印、缩印或扫描等复制手段保存、汇编本学位论文。保密的学位论文在解密后适用本授权书。}\\

\begin{flushright}\fangsong\zihao{-4}
	\begin{tabular}{lp{3cm}lp{3cm}}
		学位论文作者签名: & \underline{\makebox[3cm][l]{\hfill}} & 日期:& \underline{\makebox[3cm][l]{\hfill}} \\
		导 \hfill 师 \hfill 签 \hfill 名: & \underline{\makebox[3cm][l]{\hfill}} & 日期:& \underline{\makebox[3cm][l]{\hfill}} \\
	\end{tabular}
\end{flushright}


%%% 说明: 此部分需要自己填写的内容:  (1) 中文摘要及关键词 (2) 英文摘要及关键词
%%% ------------ 中文摘要 ---------------%%%
\begin{cnabstract}
高同学,这里是摘要,是摘要,是中文摘要,药,药,切克闹,药,药,不要停……

\end{cnabstract}
\vfill
\cnkeywords{关键词1, 关键词2, 关键词3}\\[14pt]

%%% ------------ 英文摘要 ---------------%%%

\begin{enabstract}
This thesis is a study on the theory of \dots.

\end{enabstract}
\vfill
\enkeywords{keyword1, keyword2, keyword3}\\[14pt]
    % 加入英文封面
% !Mode:: "TeX:UTF-8"

%%% 说明: 此部分需要自己填写的内容:  (1) 中文摘要及关键词 (2) 英文摘要及关键词
%%% ------------ 中文摘要 ---------------%%%
\begin{cnabstract}
高淑红同学,这里就是你的中文摘要啦

\end{cnabstract}
\vfill
\cnkeywords{关键词1, 关键词2, 关键词3}\\[14pt]

%%% ------------ 英文摘要 ---------------%%%

\begin{enabstract}
This thesis is a study on the theory of \dots.

\end{enabstract}
\vfill
\enkeywords{keyword1, keyword2, keyword3}\\[14pt]      % 加入中英文摘要.

\pdfbookmark[0]{目录}{toc}
\tableofcontents
\thispagestyle{empty}

\frontmatter  % 前言部分

\mainmatter   % 正文部分

\baselineskip=21pt  % 正文行距为 20 磅

\chapter{绪论(前言或引言)}

\section{具体使用步骤}

步骤说明
1. xxxx
2. xxxxx
3. xxxx

\section{第二步骤}

首先我们咋样

\chapter{啊啊啊}

\section{嗯嗯嗯}

\section{呃呃呃}

\chapter{嘻嘻嘻}

\section{呃呃呃}

\section{嗯嗯嗯}

\chapter{小结}

%%%=== 参考文献 ========%%%
\cleardoublepage\phantomsection
\addcontentsline{toc}{chapter}{参考文献}
\begin{thebibliography}{000}\zihao{5}

  \bibitem{r1} 作者, 文章题目, 期刊名, 年份(期数): 起止页码

  \bibitem{r2} 作者, 书名, 年份, 版次, 出版地: 出版单位, 起止页码

  \bibitem{r3} 邓建松等, 《\LaTeXe~科技排版指南》, 科学出版社

  \bibitem{r4} 吴凌云, 《CTeX~FAQ (常见问题集)》, \textit{Version~0.4}, June 21, 2004

  \bibitem{r5} Herbert Vo\ss, Mathmode, \url{http://www.tex.ac.uk/ctan/info/math/voss/mathmode/Mathmode.pdf}.

\end{thebibliography}



\backmatter
% !Mode:: "TeX:UTF-8"

%%% 此部分内容:  (1) 致谢  (2) 武汉大学学位论文使用授权协议书(无需改动)

%%%%%%%%%%%%%%%%%%%%%%%
%%% --------------- 致谢 ------------- - %%%
%%%%%%%%%%%%%%%%%%%%%%%
\acknowledgement


感谢你, 感谢他和她, 感谢大家.







%%%%%---武汉大学学位论文使用授权协议书---%%%%%%%%%%%%
%%%%%%%%%%%%%%%%%%%%%%%%%%%%%%%%%%%
%%%%%%%%%%%%%%%%%%%%%%%%%%%%%%%%%%%
\cleardoublepage
\newpage\vspace*{20pt}
\begin{center}{\zihao{-2}\heiti 武汉大学学位论文使用授权协议书}\end{center}
\par\vspace*{30pt}

本学位论文作者愿意遵守武汉大学关于保存、使用学位论文的管理办法及规定,
即:学校有权保存学位论文的印刷本和电子版, 并提供文献检索与阅览服务;
学校可以采用影印、缩印、数字化或其它复制手段保存论文;
在以教学与科研服务为目的前提下, 学校可以在校园网内公布部分及全部内容.
\begin{enumerate}[1、]
  \item  在本论文提交当年, 同意在校园网内以及中国高等教育文献保障系
           统(CALIS)高校学位论文系统提供查询及前十六页浏览服务.
  \item  在本论文提交~$\Box$~当年/~$\Box$~一年/~$\Box$~两年
            /~$\Box$~三年/~$\Box$~五年以后, 同意在校园网内允许读者
            在线浏览并下载全文, 学校可以为存在馆际合作关系的兄弟高校用
            户提供文献传递服务和交换服务.(保密论文解密后遵守此规定)
\end{enumerate}

\vskip 15mm

论文作者(签名):\raisebox{-1ex}{\underline{\makebox[5cm][c]{}}}
\vskip2em
				          				
学\qquad\qquad\quad 号:\raisebox{-1ex}{\underline{\makebox[5cm][c]{}}}
\vskip2em	
					
学\qquad\qquad\quad 院:\raisebox{-1ex}{\underline{\makebox[5cm][c]{}}}					

\vskip  2cm
\begin{flushright}
 日期:\hskip2cm 年\hskip1.2cm 月\hskip1.2cm 日
\end{flushright}

%%%%%%%%%%%%%%%%%%%%%%%%%%%%%%%%%%%%%%%
%%%%%%%--判断是否需要空白页-----------------------------
  \iflib
  \else
  \newpage
  \cleardoublepage
  \fi
%%%%%%%-------------------------------------------------







 %%%致谢, 武汉大学学位论文使用授权协议书.

\cleardoublepage
\end{document}
