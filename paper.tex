% !TEX encoding = UTF-8 Unicode
\documentclass[c5size,a4paper,hyperref,fancyhdr,UTF8]{ctexart}

% 使用 word 的 A4 版面设置
\usepackage[top=3cm,bottom=2cm,left=3cm,right=2.5cm]{geometry}

% 设置列表
\usepackage{enumitem}
\setlist[enumerate, 1]{label=(\arabic*),leftmargin=4em,parsep=\parskip}

% 数学和定理环境设置
\usepackage[all]{xy}
\usepackage{amssymb,amsmath}
\usepackage[amsmath,thref,thmmarks]{ntheorem}
{
  \theoremstyle{nonumberplain}
  \theoremheaderfont{\indent\bfseries}
  \theorembodyfont{\normalfont}
  {
    \theoremsymbol{}
    \newtheorem{proof}{证明}
    \newtheorem{solution}{解}
  }
  {
    \theoremsymbol{}
    \newtheorem{notation}{注}
  }
}

\theoremheaderfont{\indent\heiti}
\theorembodyfont{\kaishu}
\theoremnumbering{arabic}
\theoremseparator{\hspace{.5\ccwd}}

\makeatletter
\renewtheoremstyle{plain}%
  {\item[\hskip\labelsep \theorem@headerfont ##1\ ##2\theorem@separator]}
  {\item[\hskip\labelsep \theorem@headerfont ##1\ ##2\ ##3\theorem@separator]} %引用不要 ()
\makeatother

\newtheorem{definition}{定义}[section]
\newtheorem{axiom}{公理}
\newtheorem{theorem}[definition]{定理}
\newtheorem{proposition}[definition]{命题}
\newtheorem{corollary}{推论}[section]
\newtheorem{lemma}{引理}[section]
\newtheorem{exercise}{习题}[section]
{%单独为例设置宋体字体
\theorembodyfont{\songti}
\newtheorem{example}[definition]{例}
}

\iffalse
\theoremstyle{nonumberplain}
\theoremsymbol{}
\newtheorem{proof}{证}
\newtheorem{solution}{解}
\theoremsymbol{}
\newtheorem{notation}{注}
\fi

\setlength{\theorempreskipamount}{0.2cm}
\setlength{\theorempostskipamount}{0.2cm}

\hyphenation{TeX ACM}

% 设置上标引用
\newcommand{\upcite}[1]{\textsuperscript{\cite{#1}}}

% 设置页眉页脚
\usepackage{fancyhdr}
\pagestyle{plain}

% 设置标题样式
\CTEXsetup[format+={\raggedright},titleformat+={\songti}]{section}

% 参考文献设置
\CTEXoptions[bibname={\zihao{5}\songti\centerline{参考文献}}]
\bibliographystyle{abbrv}

% 行距和断后空格设置
\linespread{1.3}

\title{ \songti\textbf{相容连续范畴}
    \thanks{ \textbf{基金项目:}国家自然科学基金资助项目(11171156), 安徽省自然科学研究项目(KJ2012Z358) }}
\author{ \songti\textbf{朱润秋,卢涛,马晶晶}\\
    (淮北师范大学 数学科学学院,安徽 淮北  235000)
    \thanks { \textbf{作者简介:}朱润秋(1988-),女,安徽宿州人,硕士生,主要从事格上拓扑学理论的研究 }}

\date{}

\begin{document}

\maketitle

\begin{abstract}
\zihao{5}
\noindent \textbf{摘要: }本文给出了相容定向完备范畴的概念及在此结构下的~way-below~关系与连续性, 讨论了相
容定向完备范畴上的辅关系特别是~way-below 关系的一些性质, 证明了在相容连续范畴中~way-below 关系满足强插值
性质. 最后给出了相容连续范畴的一个刻画定理, 即相容定向完备范畴 $L$ 是相容连续的当且仅当对任意
的 $x\in obL$, $\downarrow x$ 是连续范畴.

\noindent \textit{\textbf{关键词: }相容定向完备范畴; 连续; way-below关系; 插值性质}
\end{abstract}

\begin{abstract}
\zihao{5}
\begin{center}
  \title{\textbf{Consistently Continuous Category}}
  \\(School of Mathematical Sciences, Huaibei Normal University,
  \\Huaibei, Anhui, 235000, PR China)
\end{center}

\noindent \textbf{Abstract:} In this paper, we introduce the concepts of the consistently directed complete categories and the relation of way-below in this structure. After that a concept of continuous and the properties of the way below relation on consistently directed complete categories are discussed. It is proved that the way-below relation satisfies the strong interpolation property in a consistently continuous category. At last, it is showed that a consistently directed complete category $L$ is consistently continuous if and only if for all $x\in obL$, $\downarrow x$ is a continuous.

\noindent \textit{\textbf{keywords: }consistently directed complete category; continuous; the way below relation; interpolation property}
\end{abstract}

连续格概念是著名逻辑学家~Scott~引入的, 它在研究计算机语言时有着重要作用. 后来, 人们推广连续格概念, 将其中关键的双小于移植到偏序集上, 得到连续偏序集概念\upcite{r1,r2}. 近年来理论计算机中研究的各种~Domain~则是特殊的连续偏序集, 它们一般具有良好的局部性质.

从范畴论角度看, 一个偏序集是一个小的~thin, skeletal~范畴\upcite{r3}. 文献\cite{r4}将偏序集上的~way-below~关系推广到任意小范畴上,
在任意小范畴上引入连续性, 证明了连续范畴有许多类似于连续偏序集的好的性质.
本文借助文献\cite{r5}中提出的相容定向完备集的定义, 引入了相容定向完备范畴的概念,并将~way-below~关系转移到相容定向完备范畴上,
给出了相容连续范畴的概念. 证明了在相容连续范畴中~way-below~关系 $\Rightarrow$ 满足强插值性质.
最后给出了相容连续范畴的一个刻画定理, 即相容定向完备范畴~$L$ 是相容连续的当且仅当对任意的~$x\in obL$,$\downarrow x$ 是连续范畴.

\section{预备知识}
\label{sec:introd}

在本文中, 所涉及到的范畴都为小范畴. 记一个范畴 $L$ 中的态射为 $\rightarrow$. 对任意的 $x\in obL$,
记 $\downarrow x=\{y\in obL:y\rightarrow x\}$.

\begin{definition}[\upcite{r4}]
设 $L$ 是一个小范畴, $D:J\rightarrow L$ 是一个 $J$ 型图. 如果对任意的 $j,j'\in obJ$ 都存在 $i\in obJ$ 使得 $D(j)\rightarrow D(i)$, $D(j')\rightarrow D(i)$, 则称 $D$ 是一个定向图表.
$$
\xymatrix{
    &D(i)& \\
    D(j)\ar[ur]^{\lambda_{j}}&&D(j')\ar[ul]_{\lambda_{j'}}
}
$$
若 $L$ 中每一个定向图表 $D$ 的上确界 $colim D$ 都存在, 则称 $L$ 是一个定向完备范畴.
\end{definition}

\begin{definition}[\upcite{r4}]
设 $L$ 是一个定向完备范畴, $x,y\in obL$, 称 $x$~way-below~于$y$, 记作$x\Rightarrow y$, 当且仅当对每一个定向图表 $D$, 若存在态射 $y\rightarrow colim D$, 则总有 $d\in obL$, 使得 $x\rightarrow d$.
$$
\xymatrix{
    y\ar@{.>}[r] & colim D\\
    x\ar@{=>}[u]\ar[r] & d\ar@{.>}[u]}
$$
若对任意的$x\in obL$, $\{u\in obL:u\Rightarrow x\}$是定向的且$x=\coprod\{u\in obL:u\Rightarrow x\}$, 则称 $L$ 是一个连续范畴.
\end{definition}

\begin{proposition}[\upcite{r4}]\label{propos:1}
设$L$是一个余完备范畴, 则对任意的$u,x,y,z\in obL$下列结论成立:
\begin{enumerate}
\item 若$x\Rightarrow y$, 则$x\rightarrow y$;
\item 若$u\rightarrow x\Rightarrow y\rightarrow z$, 则$u\Rightarrow z$;
\item $0\Rightarrow x$.
\end{enumerate}
\end{proposition}

\begin{definition}[\upcite{r4}]
设$L$是一个范畴且具有初始对象, 称$obL$上的一个二元关系$\prec$为辅关系, 若对任意的 $u,x,y,z\in obL$,$\prec$ 满足下列条件:
\begin{enumerate}
\item 若 $x\prec y$ 成立, 则 $x\rightarrow y$ 成立;
\item 若 $u\rightarrow x\prec y\rightarrow z$, 则 $u\prec z$ 成立;
\item $0\prec y$.
\end{enumerate}
\end{definition}

我们记~$obL$ 上的所有的辅关系构成的集合为 $Aux(obL)$, 由命题\ref{propos:1} 可知, ~way-below~关系~$\Rightarrow$ 是一个辅关系.


\section{相容定向完备范畴}

\begin{definition}
  设 $D:C\rightarrow L$是 $L$ 中一个定向图表, 若存在 $x\in obL$ 使得 $D$ 是 $\downarrow x$ 的子图表, 则称 $D$ 为相容定向图表.
\end{definition}

\begin{definition}
  设~$L$~是小范畴, 如果 $L$ 中每一个相容定向图表 $D$ 的上确界 $colim D$ 都存在, 则称 $L$ 是相容定向完备范畴.
\end{definition}

\begin{example}\label{exp:2.3}
  实数集 $\mathbf{R}$ 和自然数集 $\mathbf{N}$ 关于通常序作为范畴都是相容定向完备范畴.
\end{example}

\begin{notation}
  定向完备范畴一定是相容定向完备范畴, 但反之不然. 如例\ref{exp:2.3}中两个范畴都是相容定向完备范畴,但它们都不是定向完备范畴, 因为它们按通常序都不是定向完备的.
\end{notation}

\begin{definition}
  设 $L$ 是一个相容定向完备范畴, $x,y\in obL$,$x$~way-below~于 $y$,记作 $x\Rightarrow y$ 当且仅当对 $L$ 中任意一个相容定向图表 $D:C\rightarrow L$, 如果存在态射 $y\rightarrow colim D$, 则总有 $d\in D(obC)$, 使得 $x\rightarrow d$ 成立.
\end{definition}

\begin{proposition}\label{prop:2.5}
  设 $L$ 是相容定向完备范畴, 则对任意的 $u,x,y,z\in obL$, 下列结论成立:
  \begin{enumerate}
    \item 若 $x\Rightarrow y$, 则$x\rightarrow y$;
    \item 若 $u\rightarrow x\Rightarrow y\rightarrow z$, 则$u\Rightarrow z$;
    \item 若 $L$ 有初始对象 $0$, 则$0\Rightarrow x$.
  \end{enumerate}
\end{proposition}

\begin{proof}
  (1) 取~$D(obC)={y}$, 若 $x\Rightarrow y$, 由定义2.4知, $x\rightarrow y$.\\
  \indent (2) 设~$u\rightarrow x\Rightarrow y\rightarrow z$, 对 $L$ 中每一个相容定向图表 $D:C\rightarrow L$, 如果存在态射 $z\rightarrow colim D$, 那么有 $y\rightarrow colimD$,因 $x\Rightarrow y$, 则存在 $d\in D(obC)$, 使得 $x\rightarrow d$, 又 $u\rightarrow x$, 则 $u\rightarrow d$, 因此 $u\Rightarrow z$.\\
  \indent (3) 对 $L$ 中每一个相容定向图表 $D:C\rightarrow L$, 如果存在态射 $x\rightarrow colim D$, 因0是初始对象, 则总有 $d\in D(obC)$, 使得 $0\rightarrow d$, 故$0\Rightarrow x$.
\end{proof}

我们记~$\Uparrow x=\{v\in obL:x\Rightarrow v\}$,$\Downarrow x=\{u\in obL:u\Rightarrow x\}$, 分别称为~$x$ 在~$L$ 中的~way-below~上集和下集.

\begin{definition}\label{def:2.6}
  设 $L$ 是相容定向完备范畴, 若对每一个 $x\in obL$, $\Downarrow x$ 是相容定向图表, 且 $x=\amalg\Downarrow x$, 则称 $L$ 是相容连续范畴.
\end{definition}

\begin{definition}\label{def:2.7}
  设 $L$ 是相容定向完备范畴, 称定义在 $obL$ 上的辅关系 $\prec$ 为逼近的, 如果 $\{u\in obL:u\prec x\}=s_{\prec}(x)$ 是相容定向的且对任意的 $x\in obL$, $x=\amalg\{u\in obL:u\prec x\}=\amalg s_{\prec}(x)$.
\end{definition}

我们记 $obL$ 上的所有逼近的辅关系构成的集合为 $App(obL)$. 由定义\ref{def:2.6}可知, 在相容连续范畴中~way-below~关系 $\Rightarrow$ 是逼近的辅关系.\vspace{0.2cm}

下面, 我们就定义在相容定向完备范畴上的辅关系及way-below关系的性质进行一些讨论.

\begin{proposition}
  在相容定向完备范畴中,~way-below~关系包含于所有逼近的辅关系.
\end{proposition}

\begin{proof}
  设 $y\Rightarrow x ,\prec\in App(obL)$, 则 $\{u\in obL:u\prec x\}$ 是相容定向图表, 且 $\forall x\in obL$, $x=\amalg\{u\in obL:u\prec x\}$. 于是, $\exists u$使得 $y\rightarrow u\prec x$, 由定义1.4(2)可得~$y\prec x$, 故~way-below~关系 $\Rightarrow$ 包含于所有逼近的辅关系
\end{proof}

\begin{proposition}\label{prop:2.9}
  设 $L$ 是相容定向完备范畴, 则 $L$ 是相容连续范畴当且仅当~way-below~关系 $\Rightarrow$ 是$obL$ 上最小的逼近辅系.
\end{proposition}

\begin{proof}
  由定义\ref{def:2.6}及定义\ref{def:2.7}可知, $L$ 是相容连续范畴当且仅当~way-below~关系 $\Rightarrow$ 是逼近的辅关系,再由命题2.8, ~way-below~关系是最小的逼近辅关系, 故结论得证.
\end{proof}

我们称小范畴 $L$ 上的辅关系 $\prec$ 满足强插值性质, 若对任意的 $x,z\in obL$ 满足条件:
\begin{center}
  若 $x\prec y$ 且 $z\nrightarrow x$ ,则存在 $y$, 使得 $x\prec y\prec z$ 且 $x\neq y$.(SI)
\end{center}
称 $\prec$ 满足插值性质, 若对任意的 $x,z\in obL$, 下面较弱条件成立:
\begin{center}
  若 $x\prec y$, 则存在 $y$, 使得 $x\prec y\prec z$.(INT)
\end{center}

易知,若条件(SI)成立则条件(INT)一定成立. 文献\cite{r4}中命题2给出, 在连续范畴中~way-below~关系 $\Rightarrow$ 满足插值性质, 那么在相容连续范畴中, 这一性质是否仍成立, 并且能否加强呢?

\begin{proposition}\label{prop:2.10}
  设 $L$ 是相容定向完备范畴, 则 $obL$ 上的任何逼近的辅关系 $\prec$ 满足下列条件:对任意的 $x,z\in obL$, \par(1)~~ 对 $L$ 中任一相容定向图表 $D:C\rightarrow L$, 若 $x\Rightarrow z, z\nrightarrow x$ 且存在态射 $z\rightarrow colim D$, 则总有 $y\in D(obC)$, 使得 $x\prec y$ 且 $x\neq y$.
  \par(2)~~若 $x\Rightarrow z$ 且 $z\nrightarrow x$, 则存在 $y$, 使得 $x\prec y\prec z$ 且 $x\neq y$.
\end{proposition}

\begin{proof}
  (1) 设 $D:C\rightarrow L$ 是 $L$ 中任一相容定向图表, 且存在态射 $z\rightarrow colim D$. 令 $I=\amalg\{s_{\prec}(d):d\in D(obC)\}$, 由辅关系 $\prec$ 是逼近的知 $s_{\prec}(d)$ 是相容定向的且 $colim I=\amalg\{\amalg s_{\prec}(d):d\in D(obC)\}=\amalg D=colim D\leftarrow z$. 因此, 若 $x\Rightarrow z$, 则 $\exists i\in I$, 使得 $x\rightarrow i$, 亦即 $\exists d\in D(obC)$, 使得 $x\rightarrow i\prec d$, 于是 $x\prec d$. 由于 $z\nrightarrow x$, 又 $z\rightarrow colim D$, 所以$\exists e\in D(obC)$, 使得  $e\nrightarrow x$. 令 $y$ 为 $d$ 与 $e$ 在 $D$ 中的上界, 则 $y$ 满足条件: $y\in D(obC)$, 使得 $x\prec y$ 且 $x\neq y$.\\
  (2)~~取 $s_{\prec}(z)=\{y\in obL:y\prec z\}$, 由于 $\prec$ 是逼近的, 则 $s_{\prec}(z)$ 是相容定向的且 $z=\amalg s_{\prec}(z)=colim D$. 如果$x\Rightarrow z$ 且 $z\nrightarrow x$, 由(1)知,可找到 $y\in D(obC)$, 使得 $x\prec y$ 且 $x\neq y$, 从而有 $x\prec y\prec z$ 且 $x\neq y$.
\end{proof}

由命题\ref{prop:2.9}及命题\ref{prop:2.10}可得如下结论:

\begin{theorem}
  在相容连续范畴中~way-below~关系 $\Rightarrow$ 满足强插值性质, 即若$x\Rightarrow z$ 且
$z\nrightarrow x$, 则存在 $y$, 使得 $x\Rightarrow y\Rightarrow z$ 且 $x\neq y$.
\end{theorem}

\section{相容定向完备范畴的一个刻画}

设~$L$ 为一个小范畴, 对任意的~$x\in obL$, 易知~$\downarrow x=\{y\in obL:y\rightarrow x\}$ 是一个小范畴.

\begin{proposition}\label{prop:3.1}
  若~$L$ 是相容连续范畴,则对任意的 $x\in obL$, $\downarrow x$ 是连续范畴且对任意的 $x\in obL, u\in ob\downarrow x$, $u$ 在$\downarrow x$ 中的~way-below~下集 $\Downarrow_{x}u$ 与 $u$ 在 $L$ 中的~way-below~ 下集 $\Downarrow u$ 相同, 即 $\Downarrow_{x}u=\Downarrow u$.
\end{proposition}

\begin{proof}\label{prop:3.2}
  设 $L$ 是相容连续范畴, $\forall x\in obL, \forall u\in ob\downarrow x$. 若 $y\Rightarrow u$, 显然有   $y\Rightarrow_{x} u$, 即 $\Downarrow u\subseteq\Downarrow_{x}u$. 又若 $y\Rightarrow_{x}u$, 由 $\Downarrow u\subseteq\Downarrow_{x}u\subseteq\downarrow x$ 知~$\Downarrow u$ 是定向的且~$u=\amalg\Downarrow u$, 则$\exists v\in\Downarrow u$, 使得 $y\rightarrow v\Rightarrow u$, 于是由命题\ref{prop:2.5}(2)知, $y\Rightarrow u$, 因 此 $\Downarrow_{x}u=\Downarrow u$. 从而 $\Downarrow_{x}u$是定向的且 $u=\amalg\Downarrow u=\amalg\Downarrow_{x}u$, 所以 $\downarrow x$ 是连续范畴,故结论得证.
\end{proof}

\begin{proposition}
  若 $L$ 是相容定向完备范畴,且对任意的 $x\in obL$, $\downarrow x$ 都是连续范畴,则 $L$ 是相容连续范畴.
\end{proposition}

\begin{proof}
    考虑任一 $x\in obL$, 只要证 $\Downarrow x=\Downarrow_{x}x$, 便可由 $\downarrow x$ 连续得
    $\Downarrow x$ 是定向的且 $x=\amalg\Downarrow x$. 事实上, 若 $y\Rightarrow x$, 显然有 $y\Rightarrow_{x}x$,
    即 $\Downarrow x\subseteq\Downarrow_{x}x$.又若 $y\Rightarrow_{x}x$ 但 $y\nRightarrow x$,
    则存在定向图表 $D:C\rightarrow L$ 使得 $colim D=z\leftarrow x$ 而 $\forall d\in D(obC), y\nrightarrow d$.
    由于 $\downarrow z$ 是连续的且 $x\in ob\downarrow z$, 则由命题3.1知, $\downarrow x$ 也是连续的且 $x$ 在 $\downarrow z$
    中的~way-below~下集 $\Downarrow_{z}x=\Downarrow_{x}x$, 故由 $y\Rightarrow_{x}x\rightarrow z$ 可得 $y\Rightarrow_{z}x$.
    又由 $colim D=z\leftarrow x$, 得 $\exists d\in D(obC)$ 使得 $y\rightarrow d$ 成立, 这与上面的假设矛盾, 所以 $y\Rightarrow x$,
    因此 $\Downarrow_{x}x=\Downarrow x$. 由$\downarrow x$ 是连续的,则 $\Downarrow_{x}x$ 是定向的且 $x=\amalg\Downarrow_{x}x$,
    从而 $\Downarrow x$ 是定向的且 $x=\amalg\Downarrow x$, 故 $L$ 是相容连续范畴.
\end{proof}

综合命题\ref{prop:3.1}和命题\ref{prop:3.2}可得如下相容连续范畴的一个刻画定理:

\begin{theorem}
    设 $L$ 是相容定向完备范畴, 则 $L$ 是相容连续范畴当且仅当对任意的 $x\in obL$, $\downarrow x$ 是连续范畴.

    $\overrightarrow{shuxue}  \Downarrow$\\
    \indent $a^{2}+b^{2}=c_{2}^{2}$\\
    \indent a   $a$\\
    \indent x   $x$
\end{theorem}

\begin{thebibliography}{000}
  %\bibitem{label1} 作者, 文章题目, 期刊名, 年份(期数): 起止页码
  %\bibitem{label2} 作者, 书名, 年份, 版次, 出版地: 出版单位, 起止页码

  \bibitem{r1} Gierz G, Hofmann KH, Keimel K, etal.Continuous Lattice and Domains [M]. Cambridge: Cambridge University Press, 2003.23-33.
  \bibitem{r2} 郑崇友,樊磊,崔宏斌.Frame与连续格.北京:首都师范大出版社,2000:1-33.
  \bibitem{r3} 贺伟,范畴论[M].北京:科学出版社,2006.
  \bibitem{r4} 卢涛,王习娟,贺伟.连续范畴[J].南京师大学报(自然科学版),2008,31(3):12-15.
  \bibitem{r5} 徐罗山.相容连续偏序集及其定向完备化[J].扬州大学学报(自然科学版),2000,3(1):1-10.
\end{thebibliography}
\end{document}
